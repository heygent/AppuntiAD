\PassOptionsToPackage{unicode=true}{hyperref} % options for packages loaded elsewhere
\PassOptionsToPackage{hyphens}{url}
%
\documentclass[italian,]{book}
\usepackage{lmodern}
\usepackage{amssymb,amsmath}
\usepackage{ifxetex,ifluatex}
\usepackage{fixltx2e} % provides \textsubscript
\ifnum 0\ifxetex 1\fi\ifluatex 1\fi=0 % if pdftex
  \usepackage[T1]{fontenc}
  \usepackage[utf8]{inputenc}
  \usepackage{textcomp} % provides euro and other symbols
\else % if luatex or xelatex
  \usepackage{unicode-math}
  \defaultfontfeatures{Ligatures=TeX,Scale=MatchLowercase}
\fi
% use upquote if available, for straight quotes in verbatim environments
\IfFileExists{upquote.sty}{\usepackage{upquote}}{}
% use microtype if available
\IfFileExists{microtype.sty}{%
\usepackage[]{microtype}
\UseMicrotypeSet[protrusion]{basicmath} % disable protrusion for tt fonts
}{}
\IfFileExists{parskip.sty}{%
\usepackage{parskip}
}{% else
\setlength{\parindent}{0pt}
\setlength{\parskip}{6pt plus 2pt minus 1pt}
}
\usepackage{hyperref}
\hypersetup{
            pdftitle={Appunti di Distributed Algorithms},
            pdfauthor={Emanuele Gentiletti},
            pdfborder={0 0 0},
            breaklinks=true}
\urlstyle{same}  % don't use monospace font for urls
\setlength{\emergencystretch}{3em}  % prevent overfull lines
\providecommand{\tightlist}{%
  \setlength{\itemsep}{0pt}\setlength{\parskip}{0pt}}
\setcounter{secnumdepth}{5}
% Redefines (sub)paragraphs to behave more like sections
\ifx\paragraph\undefined\else
\let\oldparagraph\paragraph
\renewcommand{\paragraph}[1]{\oldparagraph{#1}\mbox{}}
\fi
\ifx\subparagraph\undefined\else
\let\oldsubparagraph\subparagraph
\renewcommand{\subparagraph}[1]{\oldsubparagraph{#1}\mbox{}}
\fi

% set default figure placement to htbp
\makeatletter
\def\fps@figure{htbp}
\makeatother

\usepackage{amsthm}
\ifnum 0\ifxetex 1\fi\ifluatex 1\fi=0 % if pdftex
  \usepackage[shorthands=off,main=italian]{babel}
\else
  % load polyglossia as late as possible as it *could* call bidi if RTL lang (e.g. Hebrew or Arabic)
  \usepackage{polyglossia}
  \setmainlanguage[]{italian}
\fi

\title{Appunti di Distributed Algorithms}
\author{Emanuele Gentiletti}
\date{}

\begin{document}
\maketitle

\newcommand{\outnbrs}{out\text{-}nbrs} 
\newcommand{\innbrs}{in\text{-}nbrs}
\newcommand{\indist}[1]{\overset{#1}{\sim}}
\newtheorem{theorem}{Teorema}[chapter]

{
\setcounter{tocdepth}{2}
\tableofcontents
}
\hypertarget{introduzione}{%
\chapter{Introduzione}\label{introduzione}}

Il termine \emph{algoritmi distribuiti} include una grande varietà di
algoritmi concorrenti usati per una vasta gamma di applicazioni.
Originalmente, il termine si riferiva ad algoritmi progettati per essere
eseguiti su molti processori ``distribuiti'' su una vasta area
geografica. Nel corso degli anni, date le similarità riscontrate tra
queste categorie, il termine racchiude anche gli algoritmi eseguiti su
LAN e su multiprocessori a memoria condivisa.

Il termine ``processori'' suggerisce che si stia trattando di parti
hardware. Ai fini di questo trattato è più utile pensare a questi come
processi sofware in esecuzione sui processori hardware.

Ci sono molti tipi di algoritmi distribuiti. Alcuni degli attributi per
cui si differenziano sono:

\begin{itemize}
\item
  \emph{Il modello di comunicazione tra processi (IPC)}: Gli algoritmi
  distribuiti sono eseguiti su più processori, che hanno bisogno di
  comunicare in qualche modo.
\item
  Il \emph{modello di temporizzazione}: I processori possono essere
  considerati:

  \begin{itemize}
  \item
    \textbf{completamente sincroni}: eseguono comunicazioni e
    computazioni di pari passo.
  \item
    \textbf{completamente asincroni}, eseguono passi a velocità
    arbitrarie e in ordine arbitrario.
  \item
    \textbf{parzialmente sincroni}: i processori hanno informazioni
    parziali sulla temporizzazione degli eventi (es. i processori hanno
    dei limiti sulle loro velocità relativa o hanno accesso a un clock
    approssimativamente sincronizzato).
  \end{itemize}
\item
  Il \emph{modello di fallimento}: Si può supporre che il sistema sia
  completamente affidabile, o che l'algoritmo debba tollerare una certa
  quantità di comportamenti errati. I processi possono interrompersi,
  senza avvertimento, oppure possono esibire un più grave
  \emph{fallimento bizantino}, dove si comportano in maniera arbitraria.
\item
  I \emph{problemi affrontati}: Ad esempio allocazione di risorse,
  comunicazione, consenso tra processori distrubuiti, controllo della
  concorrenza in un database, rilevamento dei deadlock, snapshot
  globali, e implementazione di diversi tipi di oggetto.
\end{itemize}

Il comportamento degli algoritmi distribuiti è spesso difficile da
comprendere per molti fattori dovuti da un maggiore tasso di
\emph{incertezza} e \emph{indipendenza delle attività}. Alcuni dei tipi
di incertezza con cui gli algoritmi devono confrontarsi sono:

\begin{itemize}
\tightlist
\item
  numero di processori non noto
\item
  topologia di rete non nota
\item
  input indipendenti in diverse locazioni
\item
  diversi programmi in esecuzione tutti insieme, avviati in diversi
  momenti, e in operazione a diverse velocità
\item
  non determinismo del processore
\item
  tempi di arrivo dei messaggi incerti
\item
  fallimenti dei processori e delle comunicazioni
\end{itemize}

Negli algoritmi distribuiti, invece di comprendere tutto sul loro
comportamento, il meglio che si può fare spesso è capire alcune
proprietà specifiche e certe del loro comportamento.

\hypertarget{modello-di-rete-sincrona}{%
\chapter{Modello di rete sincrona}\label{modello-di-rete-sincrona}}

\hypertarget{sistemi-di-rete-sincroni}{%
\section{Sistemi di rete sincroni}\label{sistemi-di-rete-sincroni}}

Un \emph{sistema di rete sincrono} consiste di una collezione di
processi locati ai nodi di un grafo orientato di rete.

Per definire un sistemda di rete sincrono formalmente, si inizia con un
grafo diretto \(G = (V,E)\). Si usa la lettera \(n\) per indicare
\(|V|\), il numero di nodi nel grafo di rete.

Per ogni nodo \(i\) di \(G\), si usano le notazioni:

\begin{itemize}
\tightlist
\item
  \(\outnbrs_i\) per indicare i \emph{vicini in uscita} di \(i\), ovvero
  quei nodi che hanno archi entranti che partono da \(i\)
\item
  \(\innbrs_i\) per indicare i \emph{vicini in entrata} di \(i\), ovvero
  quei nodi da cui partono archi entranti in \(i\).
\item
  \(distance(i,j)\) indica la lunghezza del cammino minimo tra \(i\) e
  \(j\) in \(G\), se esiste. Altrimenti, \(distance(i,j) = \infty\)
\item
  \(diam\), il \emph{diametro}, indica la massima \(distance(i,j)\) tra
  tutte le coppie di nodi \((i,j)\)
\end{itemize}

Si suppone anche di avere un alfabeto fisso di messaggi \(M\), e si
considera \(null\) un segnaposto che indica l'assenza di un messaggio.

Associato a ogni nodo \(i \in V\), si ha un \emph{processo}, che
consiste formalmente dei seguenti componenti:

\begin{itemize}
\tightlist
\item
  \(states_i\), un (non necessariamente finito) insieme di \emph{stati}
\item
  \(start_i\), un sottoinsieme non vuoto di \(states_i\), indicato come
  gli \emph{stati iniziali}
\item
  \(msgs_i\), una \emph{funzione di generazione dei messaggi}, che
  associa \(states_i \times \outnbrs_i\) a elementi di
  \(M \cup \{null\}\).
\item
  \(trans_i\), una \emph{funzione di transizione di stato} che associa a
  \(states_i\) e a vettori di elementi \(M \cup \{null\}\) (indirizzati
  da \(\innbrs_i\)) a \(states_i\)
\end{itemize}

Ogni processo ha quindi un insieme di stati, di cui un sottoinsieme è di
stati iniziali. L'insieme di stati non deve essere finito, permettendo
quindi di modellare sistemi che includono strutture dati illimitate come
contatori.

\begin{description}
\item[Funzione di generazione dei messaggi]
specifica, per ogni stato e vicino in uscita, il messaggio (o \(null\))
che il processo \(i\) invia al vicino indicato, a partire dallo stato
dato.
\item[Funzione di transizione di stato]
specifica, per ogni stato e collezione di messaggi da tutti i vicini in
entrata, il nuovo stato verso cui \(i\) si sposta.
\item[Canale (o link)]
una locazione associata a ogni arco \((i,j)\) che può, in ogni momento,
contenere al massimo un singolo messaggio di \(M\).
\end{description}

L'esecuzione dell'intero sistema inizia con tutti i processi in stati
iniziali arbitrari, e con tutti i canali vuoti. Poi i processi, insieme
e passo per passo, eseguono i seguenti due passi:

\begin{enumerate}
\def\labelenumi{\arabic{enumi}.}
\item
  Applicano la funzione di generazione dei messaggi allo stato corrente
  per generare i messaggi da inviare ai vicini in uscita, e mettono
  questi messaggi nei canali appropriati.
\item
  Applicano la funzione di transizione di stato allo stato corrente e ai
  messaggi in entrata per ottenere il nuovo stato, rimuovendo i messaggi
  dai canali.
\end{enumerate}

La combinazione dei due passi è chiamata \emph{round}. Notare che, in
generale, non vengono imposte restrizioni sull'ammontare di computazione
che un processo deve eseguire per computare i valori delle sue funzioni
di generazione dei messaggi e di transizione di stato.

Notare anche che il modello presentato è deterministico, nel senso che
le funzioni di generazione dei messaggi e di transizione di stato hanno
valori singoli. Per cui, a partire da una particolare combinazione di
stati iniziali, la computazione si svolge in modo univoco.

\begin{description}
\item[Stati di halt]
L'halt dei process può essere tenuto in considerazione in questo modello
designando alcuni stati dei processi come \emph{stati di halt}, e
specificando che da questi stati non ci può più essere alcuna attività,
ovvero: nessun messaggio viene più generato, e l'unica transizione
possibile è un loop verso lo stato stesso. Questi stati non svolgono lo
stesso ruolo che hanno negli automi finiti, e \emph{non} sono quindi
considerati \emph{di accettazione}, ma hanno il solo scopo di
interrompere il processo. Ciò che viene computato dal processo deve
essere determinato in un altro modo.
\item[Nodo e processo di ambiente]
Occasionalmente, si vuole considerare un sistema sincrono in cui i
processi si avviano in diversi round. Si modella questa situazione
estendendo il grafo di rete per includere uno speciale \emph{nodo di
ambiente}, che ha archi verso tutti i nodi ordinari. Lo scopo del
corrispondente \emph{processo di ambiente} è di inviare degli speciali
\emph{messaggi di wakeup} a tutti gli altri processi. Ogni stato inziale
di ognuno degli altri processi deve essere \emph{quiesciente}, che
significa che non causa la generazione di messaggi, e che può essere
cambiato a uno stato diverso solo in seguito alla ricezione di un
messaggio di wakeup dall'ambiente o da un messaggio non nullo da un
altro precesso. Per cui, un processo può essere svegliato:
\end{description}

\begin{itemize}
\tightlist
\item
  \textbf{direttamente}, da un messaggio di wakeup dall'ambiente
\item
  \textbf{indirettamente}, da un messaggio non nullo da un altro
  processo.
\end{itemize}

\begin{description}
\item[Grafi non orientati]
A volte si vuole considerare il caso in cui il grafo di rete non è
diretto. Questa situazione viene modellata considerando un grafo diretto
con archi bidirezionali tra tutte le coppie di vicini. In questo caso,
si usa la notazione \(nbrs_i\) per indicare i vicini di \(i\) nel grafo.
\end{description}

\hypertarget{fallimento}{%
\section{Fallimento}\label{fallimento}}

Si considerano diversi tipi di fallimento nei sistemi sincroni, tra cui
\emph{fallimento dei processi} e \emph{fallimento dei collegamenti}.

\hypertarget{stopping-failure}{%
\subsection*{Stopping failure}\label{stopping-failure}}
\addcontentsline{toc}{subsection}{Stopping failure}

Il processo si interrompe nel mezzo della sua esecuzione. Nei termini
del modello, il processo potrebbe fallire prima o dopo aver eseguito una
qualche istanza del Passo 1 o Passo 2 descritti prima. In aggiunta,
potrebbe interrompersi nel mezzo del Passo 1. Questo significa che il
processo potrebbe riuscire a mettere nei canali solo un sottoinsieme dei
messaggi che dovrebbe produrre. Si da per scontato che questo possa
essere un \emph{qualunque sottoinsieme} - non si considera che il
processo produce messaggi sequenzialmente e che fallisce nel mezzo della
sequenza.

\hypertarget{fallimento-bizantino}{%
\subsection*{Fallimento bizantino}\label{fallimento-bizantino}}
\addcontentsline{toc}{subsection}{Fallimento bizantino}

Il processo può generare i suoi prossimi messaggi e il suo prossimo
stato in maniera arbitraria, senza necessariamente seguire le regole
stabilite dalle funzioni di generazione dei messaggi e di transizione di
stato.

\hypertarget{fallimento-di-collegamento}{%
\subsection*{Fallimento di
collegamento}\label{fallimento-di-collegamento}}
\addcontentsline{toc}{subsection}{Fallimento di collegamento}

Il collegamento può fallire perdendo messaggi. Nei termini del modello,
un processo può tentare di inserire un messaggio nel canale durante il
Passo 1, mentre il collegamento fallace non lo registra.

\hypertarget{input-e-output}{%
\section{Input e output}\label{input-e-output}}

Si usa la convenzione di codificare gli input e gli output negli stati.
In particolare, gli input sono inseriti in variabili designate negli
stati iniziali. Il fatto che il processo possa avere diversi stati
iniziali è importante qui, perché permette di accomodare diversi input.
Di fatto, si da normalmente per scontato che l'unica fonte di
molteplicità di stati iniziali è la possibilità di avere diversi valori
di input nelle variabili di input.

\hypertarget{esecuzioni}{%
\section{Esecuzioni}\label{esecuzioni}}

Per ragionare sul comportamento di un sistema di rete sincrono, è
necessaria una notazione formale dell'esecuzione di un sistema.

\begin{itemize}
\tightlist
\item
  Un \emph{assegnazione di stato} \(C_r\) di un sistema è definita come
  l'assegnazione di uno stato a ogni processo del sistema al round
  \(r\).
\item
  Un \emph{assegnazione di messaggio} è un'assegnazione di messaggi
  (possibilmente \(null\)) a ogni canale del sistema. \(M_r\) e \(N_r\)
  rappresentano rispettivamente i messaggi inviati e ricevuti al round
  \(r\) nel sistema, e possono differire nel caso in cui ci sia perdita
  di messaggi.
\end{itemize}

Un'esecuzione del sistema è definita come una sequenza infinita

\[C_0,M_1,N_1,C_1,M_2,N_2,C_2,\ldots\]

Ci si riferisce a \(C_r\) anche come l'assegnazione di stato che avviene
al \emph{tempo r}, ovvero il punto dopo cui \(r\) round si sono
susseguiti.

Se \(\alpha\) e \(\alpha'\) sono due esecuzioni del sistema, si dice che
\(\alpha\) è \emph{indistinguibile} da \(\alpha'\) rispetto al processo
\(i\) (denotato \(\alpha \indist{i} \alpha'\)) se \(i\) ha la stessa
sequenza di stati, la stessa sequenza di messaggi in uscita e la stessa
sequenza di messaggi in entrata sia in \(\alpha\) che in \(\alpha'\). Se
le esecuzioni condividono queste caratteristiche solo fino al round
\(r\), si può dire che le esecuzioni sono \emph{indistinguibili rispetto
a i nel corso di r round}. Queste definizioni vengono estese anche alla
situazione in cui le esecuzioni da comparare sono esecuzioni di due
differenti sistemi sincroni.

\hypertarget{metodi-di-dimostrazione}{%
\section{Metodi di dimostrazione}\label{metodi-di-dimostrazione}}

\hypertarget{asserzione-di-invarianti}{%
\subsection*{Asserzione di invarianti}\label{asserzione-di-invarianti}}
\addcontentsline{toc}{subsection}{Asserzione di invarianti}

Un'asserzione di invariante è una proprietà dello stato del sistema (in
particolare, dello stato di tutti i processi) che resta vera in ogni
esecuzione, dopo ogni round. Si può definire un'asserzione che sia vera
dopo un certo numero di round. Le asserzioni di invarianti per sistemi
sincroni sono generalmente dimostrate per induzione su \(r\), il numero
di round completati, a partire da \(r = 0\).

\hypertarget{simulazioni}{%
\subsection*{Simulazioni}\label{simulazioni}}
\addcontentsline{toc}{subsection}{Simulazioni}

L'obiettivo è mostrare che un algoritmo sincrono \(A\) ``implementa'' un
altro algoritmo sincrono \(B\), nel senso che hanno lo stesso
comportamento in termini di input/output. La corrispondenza tra \(A\) e
\(B\) è espressa tramite un asserzione che mette in relazione gli stati
di \(A\) e \(B\), quando i due algoritmi vengono eseguiti con gli stessi
input e vengono eseguiti con le stesse modalità di fallimento per lo
stesso numero di round. Un'asserzione di questo tipo è conosciuta come
una \emph{relazione di simulazione}. Come per le invarianti di
asserzione, le relazioni di simulazione sono generalmente dimostrate per
induzione sul numero di round completati.

\hypertarget{misure-di-complessituxe0}{%
\section{Misure di complessità}\label{misure-di-complessituxe0}}

\hypertarget{complessituxe0-temporale}{%
\subsection*{Complessità temporale}\label{complessituxe0-temporale}}
\addcontentsline{toc}{subsection}{Complessità temporale}

Viene misurata nei termini del numero di round richiesti fino a quando
tutti i round richiesti vengono prodotti, o finché tutti i processi si
arrestano. Se il sistema permette tempi di avvio variabili, la
complessità viene misurata dal primo round in cui avviene un
\emph{wakeup}, in qualunque processo esso occorra.

Nella pratica, la complessità temporale è più importante, non solo per
gli algoritmi sincroni ma per tutti gli algoritmi distribuiti.

\hypertarget{complessituxe0-di-comunicazione}{%
\subsection*{Complessità di
comunicazione}\label{complessituxe0-di-comunicazione}}
\addcontentsline{toc}{subsection}{Complessità di comunicazione}

Viene misurata nei termini della quantità di messaggi non-nulli che
viene inviata. Occasionalmente, si tiene anche in conto del numero dei
bit nei messaggi.

La complessità di comunicazione è significativa se causa abbastanza
congestione da rallentare l'elaborazione. L'impatto del carico della
comunicazione sulla complessità temporale può accumularsi nel caso
frequente in cui vi sono più algoritmi distribuiti in esecuzione in una
stessa rete, condividendo la stessa banda. È difficile quantificare
l'impatto che i messaggi di un algoritmo hanno sulla performance di
altri algoritmi, per cui ci si limita ad analizzare (e cercare di
minimizzare) il numero di messaggi generati dagli algoritmi individuali.

\hypertarget{randomizzazione}{%
\section{Randomizzazione}\label{randomizzazione}}

Invece di richiedere che i processi siano deterministici, a volte è
utile permettere a questi di eseguire scelte casuali, basate su qualche
distribuzione probabilistica. Dal momento in cui il modello sincrono di
base non lo permette, aumentiamo il modello introducendo una nuova
\emph{funzione random} in aggiunta a quelle di generazione dei messaggi
e di transizione, per rappresentare il passo della scelta casuale.

Formalmente, si aggiunge una componente \(rand_i\) a ogni nodo \(i\)
nella descrizione dell'automa fornita in precedenza. Per ogni stato
\(s\), \(rand_i(s)\) è una distribuzione probabilistica su di un
sottoinsieme di \(states_i\).

In ogni round di esecuzione, la funzione \(rand_i\) viene usata
inizialmente per scegliere nuovi stati, e poi le funzioni \(msgs_i\) e
\(trans_i\) vengono applicate come al solito.

L'esecuzione del sistema è definita come una sequenza infinita

\[C_0,D_1,M_1,N_1,C_1,D_2,M_2,N_2,C_2,\ldots\]

dove ogni \(C_r\) e \(D_r\) sono \emph{assegnazioni di stato} e dove
ogni \(M_r\) e \(N_r\) sono \emph{assegnazioni di messaggio}. \(D_r\)
rappresenta i nuovi stati dei processi dopo le scelte casuali del round
\(r\).

Le affermazioni su cosa viene computato da un sistema randomizzato sono
solitamente probabilistici, e asseriscono che certi risultati vengono
raggiunti con almeno un certo grado di probabilità. Quando queste
affermazioni vengono fatte, si intende generalmente che queste valgano
per tutti gli input, e, in caso di sistemi soggetti a fallimento, per
tutti i pattern di fallimento.

Per modellare gli input e i pattern di fallimento, si da' per
presupposto che un'entità fittizia chiamata \emph{avversario} controlli
le scelte di input e le occorrenze dei fallimenti, e le affermazioni
probabilistiche asseriscono che il sistema si comporta in modo corretto
in competizione con ogni avversario ammissibile.

\hypertarget{elezione-del-leader-in-un-anello-sincrono}{%
\chapter{Elezione del leader in un anello
sincrono}\label{elezione-del-leader-in-un-anello-sincrono}}

\hypertarget{il-problema}{%
\section{Il problema}\label{il-problema}}

Si suppone che il grafo di rete \(G\) sia un anello che consiste di
\(n\) nodi, numerati da 1 a \(n\) in senso orario. Si conta spesso
\(\mod n\), pemettendo a 0 di essere un altro nome per il processo
\(n\), \(n + 1\) per il processo 1, e così via.

Presupposti:

\begin{itemize}
\tightlist
\item
  I processi associati con i nodi di \(G\) non conoscono i propri
  indici, o quelli dei propri vicini.
\item
  Le funzioni di generazione dei messaggi e di transizioni sono definite
  nei termini di nomi locali e relativi per i vicini.
\item
  Ogni processo è in grado di distinguere il proprio vicino in senso
  orario da quello in senso antiorario.
\end{itemize}

Il requisito è che, eventualmente, esattamente un processo dovrebbe dare
in output la decisione per cui esso è il leader, ad esempio cambiando
una componente del suo stato chiamata \emph{status} al valore
\emph{leader}.

Ci sono diverse versioni del problema:

\begin{enumerate}
\def\labelenumi{\arabic{enumi}.}
\item
  Si richiede che tutti i processi non leader diano eventualmente in
  output il fatto che non sono leader, ad esempio cambiando il proprio
  componente \emph{status} al valore \emph{non-leader}
\item
  L'anello può essere unidirezionale o bidirezionale. Se unidirezionale,
  allora ogni arco è diretto da un processo al suo vicino in senso
  orario, per cui i messaggi possono essere inviati solo in senso
  orario.
\item
  Il numero \(n\) di nodi nell'anello può essere conosciuto o
  sconosciuto.

  \begin{itemize}
  \tightlist
  \item
    Se è conosciuto, significa che i processi devono comportarsi
    correttamente solo in anelli di dimensione \(n\), e quindi che
    possono usare il valore \(n\) nei loro programmi.
  \item
    Se è sconosciuto, significa che i processi devono essere in grado di
    lavorare in anelli di diverse dimensioni, per cui non possono usare
    informazioni sulla dimensione dell'anello.
  \end{itemize}
\item
  I processi possono essere identici o distinguersi iniziando ognuno con
  un \emph{identificatore univoco (UID)} scelto da uno spazio grande e
  completamente ordinato di identificatori, come \(\mathbb{N}^+\). Si
  da' per vero che l'UID di ogni processo sia diverso da ogni altro
  nell'anello, ma che non ci siano restrizioni su quali UID possano
  apparire nell'anello. Gli identificatori possono essere ristretti per
  essere manipolati solo tramite certe operazioni, come comparazioni, o
  possono ammettere operazioni senza restrizioni.
\end{enumerate}

\hypertarget{impossibilituxe0-per-processi-identici}{%
\section{Impossibilità per processi
identici}\label{impossibilituxe0-per-processi-identici}}

Una prima e semplice osservazione è che se tutti i processi sono
identici, il problema non può essere risolto dato il modello proposto.
Questo anche nel caso in cui l'anello sia bidirezionale e la sua
dimensione è conosciuta dai processi.

\begin{theorem} 
  
  Sia $A$ un sistema di processi, $n > 1$, ordinati in un anello bidirezionale.
  Se tutti i processi in $A$ sono identici, allora $A$ non risolve il problema
  dell'elezione del leader. 

\end{theorem}

\begin{proof}
  
Si supponga che esista il sistema $A$ descritto che risolve il problema
dell'elezione del leader. Si ottiene una contraddizione.

Si può dire senza nessuna perdita di generalità che ognuno dei processi di $A$
ha esattamente uno stato iniziale. Questo perché se ogni processo avesse più
di uno stato iniziale, si potrebbe scegliere uno qualunque degli stati
iniziali e ottenere una nuova soluzione in cui ogni processo ha un solo stato
iniziale. Con questo presupposto, $A$ ha esattamente una esecuzione.

Si consideri quindi l'univoca esecuzione di $A$. È semplice verificare, per
induzione sul numero di round $r$ che sono stati eseguiti, che tutti i
processi sono in stati identici immediatamente dopo $r$ round. Per cui, se
uno dei processi dovesse raggiungere uno stato in cui *status* corrisponde a
*leader*, allora tutti gli altri processi raggiungerebbero lo stesso stato
allo stesso momento, e questo violerebbe il requisito di unicità.

\end{proof}

\end{document}
